
\chapter{Conclusion and Outlook}

In this work, we have shown that it is possible to preserve the privacy-preserving qualities of Zcash in cross-chain transfers.
More generally speaking, we provide a specification for a decentralised cross-chain transfer protocol that fully integrates with a privacy-oriented cryptocurrency.

The author hopes that this can be of use to other teams attempting to interoperate with Zcash shielded transactions and that it encourages more privacy-friendly cross-chain transfer protocols to emerge.


\section{Future Work}

Listed below are a number of points on which improvements could or should be made, some of which may have been pointed out previously in this document:
\begin{itemize}
    \item Concurrent Issue and Redeem procedures should be feasible with minor modifications.
    \item Block header verification must be adapted to the FlyClient upgrade~\cite{zipszip0221flyclient}, as well as inclusion proofs.
    \item The way issue and redeem availability is currently awarded is non-optimal and may reveal information about the transacted amount.
    Other approaches should be explored, such as the issuing chain `handing out' issue and redeem requests to vaults at random, whereupon the vault chooses whether to accept it or not, instead of users picking a vault.
    \item A swap protocol as devised in \xclaim may turn out non-trivial to design.
    \item It may prove beneficial to allow vaults to set transaction fees themselves, allowing them to fend off network congestion and at the same time incentivising competitiveness.
    \item The note encoding/challenging approach is interesting, but not very elegant.
    It may be possible after all to somehow enforce the correctness of the encoded components in a zk-SNARK instead.
    \item Finally, developing a working proof of concept would be the next step towards a full implementation.
\end{itemize}

Furthermore, \groth, the zero-knowledge proving system employed in Sapling, is set to be replaced by the more efficient Halo2~\cite{ECCreleasescodeforHalo2} system, which also eliminates the need for a trusted setup, in a future upgrade.
This will create a new liquidity pool, which in past updates has eventually resulted in older liquidity pools drying up, as pointed out by members of the Zcash development team.
Hence \zclaim imperatively needs to be adapted to Halo2 before being implemented.

This work is planned to be summarised into a paper and submitted for publication in the coming months.
Work on the project may be continued through a Zcash Open Major Grant~\cite{AboutusZcashOpenMajorGrantsZOMG}.