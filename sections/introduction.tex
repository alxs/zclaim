
\chapter{Introduction}
Since the creation of Bitcoin in 2009, thousands of cryptocurrencies and blockchains with a wide array of applications have emerged~\cite{CoinGecko}.
This fragmentation of the blockchain space has led to a high demand for interoperability, primarily in the form of asset transfers, across blockchains.
Unfortunately, the siloed design of the vast majority of blockchains makes it difficult to design generic, cross-chain trading protocols, hence this challenge has to date been solved mostly through centralised exchanges.

However, these services require trust and undermine anonymity, which is particularly problematic when exchanging assets to or from privacy-oriented cryptocurrencies.
Such projects, also termed privacy coins, are designed around the notion of payment anonymity and have built-in privacy features that obfuscate user activity.
It is also these very features that make it especially challenging to design trustless cross-chain trading protocols involving privacy coins.

\section{Interoperability}

When talking about interoperability across blockchains nowadays, most of the time one type or another of cross-chain asset transfers is meant.
Although there have been attempts at cross-chain smart contracts~\cite{GitHubtrustlesscrosssysteminteractionIon} and they have recently started being discussed in the literature~\cite{nissl2020crossblockchain}, we still have a long way to go until that extent of interoperability becomes commonplace.

The classical solution to decentralised cross-chain exchanges are Atomic Cross-Chain Swaps (ACCS)~\cite{herlihy2018accs}.
Atomic swaps allow users on two different blockchains to swap ownership of a pre-agreed amout of assets, guaranteeing that the exchange either happens in full or not at all.
Atomic swaps work, but they also present significant limitations such as requiring to establish an external communication channel between participants in order to find and agree on a swap, an asymmetrical advantage for one out of the two participants (`free-option problem'~\cite{HomerenprojectrenWikiGitHub}) and relatively high cost and long confirmation delays.

Most other protocols that have not been designed in this manner can be classified as \emph{asset migration} protocols~\cite{zamyatin2019sok}, in which an asset is moved from one blockchain to another.
Usually, this is achieved by creating a representation of the assets on the new chain while those on the original chain are locked until the process is reversed.
This representation of the locked assets on the alien chain is often referred to as a \emph{tokenised representation} of the original assets, \emph{wrapped tokens} or \emph{cryptocurrency-backed assets (\cbas)}.

Besides, asset migration protocols may be classified as relying on a trusted third party (TTP) or as being \emph{trustless}.
Usually, relying on a TTP requires sacrificing control over one's funds and privacy, while trustless protocols require an economic system on the remote chain.


\section{Related Work}
Decentralised exchanges (DEXs)~\cite{0xwhitepaper,uniswap} offer a trustless alternative to centralised cryptocurrency exchanges.
However, most DEXs do not enable cross-chain transfers, but only serve to exchange assets within the blockchain on which they are deployed.
The Ren Project~\cite{HomerenprojectrenWikiGitHub} is a generic cross-chain transfer protocol aiming to implement universal interoperability, which they define as \textquote{the ability to send any asset from any chain to any other chain for use in any application}.
This is an impressive proposition, yet it is another question whether the issue of preserving privacy in cross-chain exchanges with privacy coins will be addressed in that context.

As far as cross-chain protocols involving privacy coins are concerned, a number of projects are currently under development.
Cosmos~\cite{cosmosWhitepaper} is working on a Zcash `pegzone'~\cite{githubPegzone}, with the ultimate goal of enabling \textquote{shielded transfers from the pegzone to Zcash and vice versa}.
Details on the project are scarce, but as per this stated purpose, the Zcash pegzone aims to achieve very much the same goal as this work.

Wrapped~\cite{Wrapped} is a tokenised representation of Zcash on Ethereum, although it relies on a centralised TTP, does not support shielded Zcash and hence can by no means be considered to preserve privacy.
Similarly, the Ren Project mentioned above offers a tokenised representation of ZCash, renZEC, which does not rely on a TTP but also does not support Zcash's shielded payment scheme.

Finally, there is an ongoing effort to implement atomic swaps between Bitcoin and the privacy coin Monero~\cite{BTCXMRatomicswaps,CCSMoneroAtomicSwapsimplementationfunding}.


\section{Contributions}

We introduce \zclaim, a protocol enabling the creation of fully private \cbas backed by the privacy coin Zcash.
This work is originally based on XCLAIM, itself a framework laying out the creation of tokens on a pre-existing, smart-contract capable \emph{issuing chain} backed by assets on a \emph{backing chain}, both of them publicly auditable.

In contrast to these assumptions, we observe the case where:
\begin{enumerate}[label=(\alph*)]
  \item the backing chain is private, i.e.\ transactions leak no significant amount of information to outsiders,
  \item this anonymity shall be retained throughout the protocol and across both chains, and
  \item the definition of the issuing chain is loosened: it may be a smart-contract capable blockchain or a blockchain with appropriate functionality, built expressly to implement this protocol.
\end{enumerate}

\todo{Outline}