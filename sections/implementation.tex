
\chapter{Case study - ZCash and Polkadot}

\cite{stewart2020grandpa} but maybe somewhere else
\todo{mention Zokrates}

We outline the theoretically abstract protocol described until here in terms of a concrete implementation between the anonymous cryptocurrency ZCash, utilised as the backing chain, and Polkadot, a sharded blockchain allowing different chains to communicate and operate together, of which one shard will serve as the issuing chain.

In particular, we describe the creation of a new token named PolkaZEC or PZEC backed 1-to-1 by ZCash funds (ZEC).
The value of 1 PZEC is always the same as that of 1 ZEC, as these are directly redeemable by the latter.

DOTs, Polkadot's native currency, will be used for collateral purposes and as a secondary means for users to obtain PolkaZEC (provided that someone is willing to swap their PZEC by DOTS).

\section{Polkadot}
\label{sec:polkadot}

Polkadot is a sharded blockchain allowing different chains to communicate between one another.
These shards or so-called parachains can either interoperate solely with other shards on the Polkadot network or also serve as a bridge between the network and an exterior blockchain, as in our case.

In the context of this work, we thus design a Polkadot parachain that serves as a bridge between ZCash and the Polkadot network.
Since the Polkadot ecosystem is still in its early stages, some components referred to in this work are still in development or planned to be implemented in the future.
Alternatively, they would be implemented as part of the bridge (though they would also be useful for other parachains due to the modular nature of Polkadot).

\todo[inline]{describe how polkadot is built (relay chain, substrate, parachains, pallets)}

\todo[inline]{describe the components we need and where they stand, is centrifuge really working on this}

For instance, we assume that Polkadot implements the same cryptographic libraries as ZCash.
We also assume that there is a mechanism in place to swap DOTs with tokens created by a parachain.


In theory, we may assume complete implementation freedom on Polkadot, the only constraint being the assumption made that the note system and cryptographic libraries from ZCash are carried over to the parachain.

\section{Implementation}
How the issuing-side components listed in the ZCLAIM Setup section can be implemented on Polkadot (not an actual implementation). Concepts (substrate, parachain, bridge, pallets, XCMP, currencies), libraries.

\section{Cost analysis}
Estimations of the cost of issuing and redeeming on Polkadot. Costs of storing and verifying ZCash block headers.


    \todo{how does collateral work on Polkadot, addresses, etc}
