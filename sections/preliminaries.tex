\chapter{Preliminaries}
\label{ch:prelims}


\begin{itemize}
    \item Concepts are usually \emph{emphasised} when first introduced.
    \item Most variables are typeset in \zcashvar{sans\text{-}serif}, though some may be in $\mathit{cursive}$.
    There is no strict distinction between the two.
    \item \textbf{Bold letters.} Bold letters are usually followed by a definition and indicate the name of something that may be referenced later.
    \item
    \begin{alg}
        \item This is a set of instructions.
        \item This is the next step.
    \end{alg}
\end{itemize}

\section{Notation}
Much of the notation in this work is borrowed from the Zcash protocol specification~\cite[Section 2]{hopwood2016zcash}.
Definitions are reproduced here for ease of reading together with the rest of notation:

\begin{tabularx}{\textwidth}{p{.13\textwidth} | p{.82\textwidth}}
    $\B$                &the type of bit values, i.e.\ \{0,\,1\}\\
    $\By$               &the type of byte values, i.e.\ \{0\,..\,255\}\\
    $\N$                &the type of nonnegative integers\\
    $\R$                &the type of real numbers\\
    $\zcashvar{x} : T$  &variable \zcashvar{x} is of type T\\
    $S \xrightarrow{\text{R}} T$  &the type of a randomised algorithm\\
    $x \xleftarrow{\text{R}} f(s)$  &sampling a variable from the output of $f$ applied to $s$, given $f : S \xrightarrow{\text{R}} T$ and $s : S$\\
    $f_x(y)$            &$f(x, y)$\\
    $T^{[\ell]}$        &the set of sequences of $\ell$ elements of type $T$\\
    $\text{``\texttt{string}''}$       &the string `\texttt{string}' represented as a sequence of bytes in US-ASCII\\
    $\{a\,..\,b\}$      &the set or type of integers from $a$ through $b$ inclusive\\
    $a\,||\,b$              &the concatenation of sequences $a$ then $b$\\
\end{tabularx}

\begin{tabularx}{\textwidth}{p{.13\textwidth} | p{.82\textwidth}}
    $\F_n$              &the finite field with $n$ elements\\
    $[k] P$             &scalar multiplication in a group as defined in~\cite[Section 4.1.8]{hopwood2016zcash}\\
    $x\star$            &bit-sequence representation of $x$, where $x$ is a group element\\
    $\&$                &bitwise AND\\
    $\bot$              &unavailable information or failure\\
    $\top$              &success\\
\end{tabularx}

\section{Sapling functions}
Below is a list of Sapling functions along with their section numbers in the protocol specification, for reference:

\begin{tabularx}{\textwidth}{p{.35\textwidth} | l}
    $\mcrh$             &5.4.1.3\\
    $\divhash$          &5.4.1.6\\
    $\sym$              &5.4.3\\
    $\kas$              &5.4.4.3\\
    $\kdfs$             &5.4.4.4\\
    $\reddsa$           &5.4.6\\
    $\redjj$            &5.4.6\\
    $\wpc$              &5.4.7.2\\
    $\ncm$              &5.4.7.2\\
    $\vcm$              &5.4.7.3\\
    $\extractj$         &5.4.8.4\\
    $\fgh$              &5.4.8.5
\end{tabularx}

\section{Sapling constants}
The following Sapling constants that appear in this document are defined in~\cite[Section 5.3]{hopwood2016zcash} and are also reproduced here for convenience:
\begin{flalign*}
    &\mds       := 32 &\\
    &\lval      := 64 &\\
    &\lms       := 255&\\
    &\ld        := 88 &\\
    &\livk      := 251&\\
    &\lscalar   := 252&
\end{flalign*}

The constants \rj, \hj, \lj, variables $\J$, $\J^{(r)}$, $\mathcal{O}_\J$ and the function $\reprj$ are defined in the context of the \zcashvar{Jubjub} curve in \cite[Section 5.4.8.3]{hopwood2016zcash}.
